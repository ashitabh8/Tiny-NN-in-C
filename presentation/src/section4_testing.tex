% ============================================================================
% SECTION 4: RUNNING THE FULL TEST SUITE
% ============================================================================

\section{Testing \& Results}

% ----------------------------------------------------------------------------
% Slide 4.1: Test Commands Summary
% ----------------------------------------------------------------------------
\begin{frame}[fragile]{Test Commands Summary}
\begin{lstlisting}[style=bashstyle]
# All tests (64 tests)
python -m pytest test/ -v

# Float model tests
python -m pytest test/test_integration.py -v -s

# Quantization unit tests
python -m pytest test/test_quantization.py -v -s

# Quantization end-to-end tests
python -m pytest test/test_quantization_e2e.py -v -s

# Optimization pass tests
python -m pytest test/test_passes.py -v -s
\end{lstlisting}

Commands for commit 9362ae0
\end{frame}

% ----------------------------------------------------------------------------
% Slide 4.2: Key Numerical Checks
% ----------------------------------------------------------------------------
\begin{frame}{Key Numerical Checks}
    \centering
    \begin{tabular}{lcc}
        \toprule
        \textbf{Test} & \textbf{Max Error} & \textbf{Test File} \\
        \midrule
        \uncover<2->{Float MLP} & \uncover<2->{$\sim 10^{-6}$} & \uncover<2->{\texttt{test\_integration.py}} \\
        \uncover<3->{Float ResNet} & \uncover<3->{$1.19 \times 10^{-7}$} & \uncover<3->{\texttt{test\_quantization\_e2e.py}} \\
        \uncover<4->{Static int8} & \uncover<4->{$\sim$1--2\%} & \uncover<4->{\texttt{test\_quantization\_e2e.py}} \\
        \uncover<5->{Static int16} & \uncover<5->{$\sim$0.1\%} & \uncover<5->{\texttt{test\_quantization\_e2e.py}} \\
        \uncover<6->{FusePass} & \uncover<6->{\textbf{0.00e+00}} & \uncover<6->{\texttt{test\_passes.py}} \\
        \bottomrule
    \end{tabular}
    
    \vspace{1em}
    
    \uncover<7->{
        \begin{alertblock}{FusePass is Bit-Identical!}
            When scales match, the optimization produces exactly the same numerical results.
        \end{alertblock}
    }
\end{frame}

% ----------------------------------------------------------------------------
% Slide 4.3: File Structure
% ----------------------------------------------------------------------------
\begin{frame}[fragile]{Project File Structure}
\begin{lstlisting}[style=diagramstyle,basicstyle=\ttfamily\tiny\color{codefg}]
src/
  pytorch_to_c/
    compiler.py              # Main entry point
    frontend/fx_tracer.py    # torch.fx tracing
    lowering/lower.py        # FX -> IR conversion
    ir/
      node.py                # IRNode base class
      graph.py               # IRGraph
      quant_node.py          # QuantIRNode base
    codegen/c_printer.py     # C code generation
    quantization/
      rules.py               # QuantRule, StaticQuantRule, etc.
      graph_transform.py     # QuantizationTransform
      ops/                   # Quantized operation implementations
  passes/
    base.py                  # IRPass base class
    fuse_dequant_quant.py    # FuseDequantQuantPass
  c_ops/
    nn_ops_float.h           # Float C kernels
    nn_ops_int8.h            # Int8 C kernels
    nn_ops_int16.h           # Int16 C kernels
\end{lstlisting}
\end{frame}

% ----------------------------------------------------------------------------
% Slide 4.4: Summary
% ----------------------------------------------------------------------------
\begin{frame}{Summary}
    \textbf{What we built:}
    \begin{itemize}[<+->]
        \item PyTorch $\rightarrow$ C compiler with torch.fx frontend
        \item Clean IR with doubly-linked node graph
        \item Independent from source rule-based quantization 
        \item Support for int8 and int16
        \item Mixed precision quantization
        \item Optimization pass infrastructure
        \item FuseDequantQuantPass (bit-identical results!)
    \end{itemize}
\end{frame}

% ----------------------------------------------------------------------------
% Slide 4.5: Extensibility Points
% ----------------------------------------------------------------------------
\begin{frame}{Extensibility Points}
    \textbf{How to extend the compiler:}
    
    \vspace{1em}
    
    \begin{columns}
    \begin{column}{0.5\textwidth}
        \textbf{Add new quantization:}
        \begin{itemize}
            \item New \texttt{QuantRule} subclass
            \item New \texttt{QuantIRNode} implementation
        \end{itemize}
        
        \vspace{1em}
        
        \textbf{Add new optimization:}
        \begin{itemize}
            \item New \texttt{IRPass} subclass
            \item Add to pass pipeline
        \end{itemize}
    \end{column}
    \begin{column}{0.5\textwidth}
        \textbf{Add new operations:}
        \begin{itemize}
            \item Add to lowering rules
            \item Add C kernel implementation
        \end{itemize}
        
        \vspace{1em}
        
        \textbf{Add new datatypes:}
        \begin{itemize}
            \item Extend \texttt{get\_c\_dtype()}
            \item Add C kernels for dtype
        \end{itemize}
    \end{column}
    \end{columns}
    
    \vspace{1em}
    
    \centering
    \textbf{Run all 64 tests:} \texttt{python -m pytest test/ -v}
\end{frame}


